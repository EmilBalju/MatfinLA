\documentclass[compress]{beamer}
\usetheme{sthlm}

\usepackage{
booktabs,
datetime,
dtk-logos,
graphicx,
multicol,
pgfplots,
ragged2e,
tabularx,
tikz,
wasysym
}
\pgfplotsset{compat=1.8}

\usepackage[utf8]{inputenc}
\usepackage{newpxtext,newpxmath}

\usepackage{listings}
\lstset{ %
language=[LaTeX]TeX,
basicstyle=\normalsize\ttfamily,
keywordstyle=,
numbers=left,
numberstyle=\tiny\ttfamily,
stepnumber=1,
showspaces=false,
showstringspaces=false,
showtabs=false,
breaklines=true,
frame=tb,
framerule=0.5pt,
tabsize=4,
framexleftmargin=0.5em,
framexrightmargin=0.5em,
xleftmargin=0.5em,
xrightmargin=0.5em
}

\usetikzlibrary{
backgrounds,
mindmap
}

\setbeameroption{show notes}

\title{Ejercicios de \\ matemáticas  \\financieras}
\subtitle{Equipo 1}

\hypersetup{
pdfauthor = {Ami}
pdfsubject = {Mathematics, },
pdfkeywords = {},  
pdfmoddate= {D:\pdfdate},          
pdfcreator = {}
}
\date{27 de Octubre del 2020}
%\title{Ejercicios de \\
%Matemáticas \\ Financieras}
\institute{Universidad Autónoma de Yucatan}
%\logo{\includegraphics[width=2.0cm]{fig/Univ}}
%\titlegraphic{\includegraphics[width=2.0cm]{fig/Univ}}
\begin{document}
%\setbeamercovered{transparent} 
%\setbeamertemplate{navigation symbols}{} 
%\logo{} 
%\institute{} 
%\date{} 
%\subject{} 


\begin{frame}
\titlepage
\end{frame}

%\begin{frame}
%\tableofcontents
%\end{frame}

\begin{frame}\frametitle{Ejercicio 1}
\begin{block}{Si $\partial (t)=0.01t$ $0\leq t \leq 2$, encuentra la tasa de interés efectiva anual equivalente sobre el intervalo $0\leq t \leq 2$.}
Sabemos que:
$$e^{\int_{0}^{n} \delta_{t} \,dt} = a(n) = (1 + i)^{n}$$
Cómo buscamos la tasa de interés efectiva anual, estamos buscando el valor de i. Así que esto lo podemos ver de la siguiente manera:
$$e^{\int_{0}^{2} 0.01t \,dt} = e^{1/50} = 1.0202$$
Entonces
$$1.0202 = (1 + i)^{2}$$
Así que:
$$\sqrt{1.0202} = (1 + i)$$
$$1.01 - 1 = 0.01 = i$$
Nuestra tasa de interés efectiva anual equivalente es de 0.01
\end{block}
\end{frame}

\begin{frame}\frametitle{Ejercicio 2}
\begin{block}{La fuerza de interés al tiempo de t es $\dfrac{t^{3}}{100}$. Encuentra $a^{-1}(3)$.}
Sabemos por información previa que $a^{-1}(t)$ es conocida como la función descuento, y para nuestro caso $a^{-1}(t) = \dfrac{1}{(1 + i)^{t}}$. \\ Eso nos da a entender que $a^{-1}(3) = \dfrac{1}{(1 + i)^{3}}$, el único valor que nos haría falta es i, pero eso lo podemos obtener a partir de la fuerza de interés.
Sabemos que:
$$e^{\int_{0}^{n} \delta_{t} \,dt} = a(n) = (1 + i)^{n}$$
Cómo estamos buscando el valor de i, esto lo podemos ver de la siguiente manera:
\end{block}
\end{frame} 

\begin{frame}\frametitle{Ejercicio 2}
\begin{block}{Continuación}
$$e^{\int_{0}^{3} \dfrac{t^{3}}{100} \,dt} = e^{81/400} = 1.2244$$
Entonces
$$1.2244 = (1 + i)^{3}$$
Así que:
$$\sqrt[3]{1.2244} = (1 + i)$$
$$1.0698 - 1 = 0.0698 = i$$
Por lo que nuestro valor de $i=0.0698$, así que nuestra $a^{-1}(3)$ es:
$$a^{-1}(3) = \dfrac{1}{(1 + 0.0698)^{3}} = 0.8167$$
\end{block}
\end{frame} 

\begin{frame}\frametitle{Ejercicio 3}
\begin{block}{Encuentra la tasa anual promedio de interés efectiva al inicio de los 3 años el cual es equivalente a una tasa de descuento efectiva de $8\%$ el primer año, $7\%$ el segundo y $6\%$ el tercero}
La tasa anual promedio de interés efectiva, lo podemos ver de la siguiente manera:
$$i= \dfrac{d_{1}+ d_{2}+ d_{3}}{3}$$
Esto es igual a:
$$i= \dfrac{[\dfrac{A(1)-A(0)}{A(1)} + \dfrac{A(2)-A(1)}{A(2)} + \dfrac{A(3)-A(2)}{A(3)}]}{3}$$
\end{block}
\end{frame}
\begin{frame}\frametitle{Ejercicio 3}
\begin{block}{Continuación}
Esto es igual a:
$$i= \dfrac{[\dfrac{1.08-1}{1.08} + \dfrac{1.07-1}{1.07} + \dfrac{1.06-1}{1.06}]}{3}$$
$$i= \dfrac{[\dfrac{0.08}{1.08} + \dfrac{0.07}{1.07} + \dfrac{0.06}{1.06}]}{3}$$
$$i = \dfrac{0.1960}{3} = 0.0653$$
\end{block}
\end{frame} 
\end{document}
